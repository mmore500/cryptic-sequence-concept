TODO paste in results

\section{Additive Effect Sites}

This assay uses ``dosed'' knockouts as the basis for estimation.
For these knockouts, we assume that we've already tperformed a single-site knockout screen to eliminate any non-cryptic sites from consideration.
Among remaining sites, we randomly sample $n$ and knock them out together.
You can easily imagine a dose-response curve where small doses would infrequently create detectable effects and large doses would more frequently create detectable effects.
The shape of this dose curve will depend on the underlying quantity of small effect sites and their average magnitude.
We propose to model this according to a negative binomial distribution, which models the number of coin flips that succed with probability $p$ needed to reach a number of successes $n$.
By fitting a negative binomial curve to estimate $n$ and $p$ we can estimate the mean fitness effect of additive sites ($1/n$ where 1 is the threshold of detectability) and the number of sites as $m \times p$ where $m$ is the genome length.

\section{Epistatic Effect Sites}

This assay uses minimal viable genomes, ``skeletons,'' as the basis for estimation.
Repeated stochastic successive knockout process until no more sites can be removed to generate a sample of skeletons.
The intuition is that sites that appear infrequently in skeletons but not others are epistatic.
However, additive small-effect sites will also appear sometimes appear outside skeletons.

We can use knockout effect size to differentiate these cases.
For this procedure, we iterate through the skeleton and do ``jackknife'' single site knockouts.
If the marginal knockout effect size of this site is very small, then it is probably an additive site.
If the marginal knockout effect of the site in the skeleton context is big, it is probably an epistatic effect site --- with the rest of its redundant sites already excluded from the skeleton.
Figure TODO shows this distinction.

\section{Any Effect Sites}

Skeletons can be put to another use to try to directly estimate the number of sites that have any fitness effect.
Reframed, skeletoniz.
Any site that has a fitness benefit should, in principle, potentially appear within a skeleton.
Skeletons randomly sample from among these sites.
Framed this way, sampling a skeleton genome is not unlike a trap sampling study used by wildlife biologists.
Extensive, well-developed mathematics exists around this problem.
Such math will need to be careful to account for ``shyness'' --- uneven probability for some sites appearing in a skeleton.
We propose to use the Burnham-Overton estimation procedure.
We give it a distribution of, among the sites that appeared within skeletons, the number of skeletons they appeared in and it gives us a total population estimate

% # Burnham, Kenneth P., and W. Scott Overton.
% # "Robust estimation of population size when capture probabilities vary among
% # animals." Ecology 60.5 (1979): 927-936.
% # https://doi.org/10.2307/1936861
