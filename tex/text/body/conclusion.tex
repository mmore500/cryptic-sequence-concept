\section{Conclusion} \label{sec:conclusion}

Much work remains in developing proposed assays for cryptic sequence complexity.
It will be particularly critical to develop strategies to navigate stochastic aspects of implicit fitness assays, which was not treated here.
Another priority is organization of assays to take advantage of parallel processing power, particulary with respect to sequential operations like pruning out genome sites to produce a minimal viable skeleton.
The feedback-dependent workflows necessary to conduct these assays will likely necessite development of software tools to orchestrate knockout trials and colate their results.
Future work should also assess statistical questions of how best to select maximally informative knockout targets and doses, as well as how bootstrapping or other procedures should be incorporated to provide confidence intervals for estimates of cryptic complexity.
% We will need methodologies to use parallel trials to rapidly strip down skeletons.
% And how to sequence knockouts performed for maximum statistical power.
Finally, building off initial, cursory demonstrations here, methods will need to be subjected to thorough and rigorous testing with a variety of full-fledged artificial life systems.
We look forward to these further steps in improving methodological capability for rigorous study of complexity in artificial life systems.

% The capability of these additive and epistatic assays to give credible results in the presence of the other type of effect needs to be studied.
% This model has obvious limitations that me  an that real-applications would be more challenging and does not reflect those challenges, the most obvious one being that the results are deterministic.
% The next steps for this project is to try with test models incorporating stochastic effects, with pre-existing alife systems that are tractable to know the underlying full structure of genomes, and with heavier, more opaque systems that this methodology is ultimately designed to help with.
% We need to figure out how to deal with stochasticity and sensitivity limitations.

 % developing and rigorously assessing the proposed methods, and collaborating with the community to put it into practice

% Unlike typical experiments with workloads that can be largely designed \textit{a priori}, divied up and sent out to run then collagted and analyzed, these experiments require a tight feeback loop where results from one stage get fed into complex statistical decision making processes to pick knockouts for the next stage.
% In order to be practically useful as a research tool, these procedures will need to be automated and to be useful across systems within the community they will need to be generalizable.
% To this end, in tandeom with statistical devleopment described, we are also developing an orchestration framework.
% Due to the dynamic back-and-forth complexity of the knockout experiments needed to perform the cryptic sequence complexity procedures  and anticipation of different computational modalities for different workloads (i.e., SLURM jobs vs local server vs. laptop).
% Because alife systems vary greatly in terms of their APIs, not to mention underlying content semantics, we have designed this system around a containerization framework.
% Under this framework, users would package their system within a container (i.e., Singularity/Docker) that implements a designated API to count available sites, knock out the $n$th site, and perform a fitness assay (e.g., competing a variant against wild type).
% A goal of this project is to host an orchestration server as a web service that can be used freely by the community without any set up.

% The container would then be hosted on a free repository like the GitHub container registry or DockerHub and the orchestration framework will do the rest --- dispatching work via container calls and receiving results via a MongoDB server.
% The system is also intended to be fully compatible with local operaiton, where the container would be located locally, the MongoDB database would be run locally, and the orchestration server would be hosted locally.

% These experiments can run on a cluster or on a users local machine.
% An under-development live API demo can be viewed at \url{http://knockem.online/api/v1/ui/}.
% Note that API keys, available by request, are necessary to interact with the system.
