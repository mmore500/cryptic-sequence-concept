\begin{abstract}
Many artificial life systems have advantages due to their inherent interprability, where you can directly assess the meaning --- and the fitness relevancy --- of genetic sites.
However, some systems for work on important topics like open-ended evolution, major transitions in evolution, and the interplay of biotic selection in ecologies are inherently more opaque and implicit.
At the extremum of these systems, it becomes necessary to analyze genome content by studying in situ fitness effects of knockout variants.
These tests, particularly when they must be performed as variant-versus-variant competitions, have inherent sensitivity limitations which mean that many sites are cryptic sequence complexity --- they contribute to fitness may not be detectable when individually knocked out.
This may be due to small additive effects or to epistatic interactions (e.g., redundancy).
Here, we propose three knockout assay designs to perform apprximate estimations of cryptic sites with additive small effects, .
We report initial trials validating on a simple genome model with explicitly configurable content to test the recoverability through assays of known cryptic sequence abundnces.
Intial results with this simple model are promising, and we look forward to more rigorous tests of these assays in upcoming work.
Ultimately, we hope to release methodology and accompanying tools that will help the community be able to more informatively study key questions about the evolution of complexity within sophisticated, intricate model systems wiht rich implicit dynamics.
\end{abstract}
