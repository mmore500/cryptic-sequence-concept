\begin{abstract}
Complexity is a signature quality of interest in artificial life systems.
Alongside other dimensions of assessment, it is common to quantify genome sites that contribute to fitness as a complexity measure.
However, limitations to the sensitivity of fitness assays in models with implicit replication criteria involving rich biotic interactions introduce the possibility of difficult-to-detect ``cryptic'' adaptive sites, which contribute small fitness effects below the threshold of individual detectability or involve epistatic redundancies.
Here, we propose three knockout-based assay procedures designed to quantify cryptic adaptive sites within digital genomes.
We report initial tests of these methods on a simple genome model with explicitly configured site fitness effects.
In these limited tests, estimation results reflect ground truth cryptic sequence complexities well.
Presented work provides initial steps toward development of new methods and software tools that improve the resolution, rigor, and tractability of complexity analyses across alife systems, particularly those requiring expensive \textit{in situ} assessments of organism fitness.

% Our ultimate aim for this work is to improve the
% through development of rigorous, tractable
% We look forward to continuing work validating and making robust proposed methods.
% Ultimately, we hope to release methodology and accompanying tools that will help the community be able to more informatively study key questions about the evolution of complexity within sophisticated, i.

% Many artificial life systems have advantages due to their inherent interprability, where you can directly assess the meaning --- and the fitness relevancy --- of genetic sites.
% However, some systems for work on important topics like open-ended evolution, major transitions in evolution, and the interplay of biotic selection in ecologies are inherently more opaque and implicit.
% At the extremum of these systems, it becomes necessary to analyze genome content by studying in situ fitness effects of knockout variants.
% These tests, particularly when they must be performed as variant-versus-variant competitions, have inherent sensitivity limitations which mean that many sites are cryptic sequence complexity --- they contribute to fitness may not be detectable when individually knocked out.
% This may be due to small additive effects or to epistatic interactions (e.g., redundancy).
\end{abstract}
