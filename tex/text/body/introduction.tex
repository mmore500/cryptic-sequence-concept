\section{Introduction} \label{sec:introduction}

Understanding under what conditions and through what mechanisms complexity evolves is a significant open question in artificial life and evolutionary biology \citep{taylor2016open,pigliucci2009extended}.
Indeed, numerous dimensions of complexity in biological systems have been considered, including those of developmental processes, phenotypic traits, and ecological interactions \citep{szathmary2001can,mcshea2000functional}.
% Any reasonable treatment of the subject should consider multiple dimensions and not try to turn it into one dimension.
Among dimensions of biological complexity, the information content of genetic sequences is often useful due to its relative tractability, generalizability across systems, and foundational role to other aspects of biological complexity \citep{adami2002complexity}.
Although information theoretic formulations have been established to describe biological sequence complexity \citep{weiss2000information}, counts of adaptive sites (i.e., genome sites that benefit fitness) are a convenient, commonly-used proxy measure for genome sequence complexity \citep{dolson2019modes}.

However, challenges can arise in identifying adaptive sites within a genome.
Beyond very small genome sizes, complete identification of adaptive sites is hindered by combinatoric effects that make all-combinations analyses necessary to fully detangle epistatic effects effectively intractable \citep{nitash2021information,adami2000evolution}.
This is particularly the case for systems with implicit fitness conditions with extensive biotic selection effects \citep{moreno2022exploring,channon2000towards}.
In such circumstances, it can become necessary to use head-to-head competition trials between wildtype strains and knockout variants \textit{in situ} to detect fitness effects \citep{moreno2021case}.
Such competition-based fitness assays typically have sensitivity limitations, which limit detection of sites with small fitness effects.
Here, we term genome sites with adaptive effects that are not directly detectable through single-site knockouts as ``cryptic'' sequence complexity.

% Multiple replicates could be performed or, alternately, wildtype vs wildtype replicates could be performed to create a null distribution and then you can see if the abundances for a variant-vs-wildtype competition fall outside that null distribution.


% All sites knockouts one-by-one can fail to detect important things.
% Another classic way to do it is an all pairs knockout.
% However, all combinations work scales exponentially with the number and isn't practical past a few sites (cite Nitash).

% Willing to trade-off certainty in precisely what sites contribute to fitness for a statistical estimate of how many sites contribute to fitness and what the character of those contributions are.
% Propose three testing and statistical estimation frameworks to get at this question of cryptic sequence complexity.


% Single-instruction nopouts  reveal sites that are critical to fitness, but are insufficient to reveal all sites that contribute to fitness.
% For example, when instructions or modules are necessary but functionally redundant, disabling only one at a time will not discover their contribution to fitness.
% Indeed, we discovered that in 97 out of the 101 representative specimens analyzed, the Fitness-noncritical Nopout variant was significantly less fit than wildtype --- clearly noncritical sites also were contributing to fitness.

% Ideally, in order to whittle down to a  ``Fitness-neutral nopout'' skeleton, we would perform a Jenga-like procedure similar to that used to arrive at the ``Phenotype-neutral nopout''.
% However, because fitness is implicit in this simulation system and must be assessed statistically through competition experiments, it is costly to assess and cannot be determined with perfect reliability.
% (Playing a large game of Jenga to completion becomes intractable if attempting to remove a piece takes more than a few seconds or you can remove a critical piece early on without realizing it.)
To enable more complete sequence complexity analyses of digital organisms inclusive of ``cryptic'' adaptive sites, we propose three assays to conduct statistical estimates of
\begin{enumerate}
\item \textbf{Additive-effect cryptic sites:} sites with small contributions to fitness that, individually, fall below the threshold of detectability of fitness assays,
\item \textbf{Epistatic-effect cryptic sites:} sites with fitness effects that are only observable in the context of other knockouts (i.e., redundancy),
\item \textbf{Any-effect cryptic sites:} sites with any contribution to fitness, inclusive of the above.
\end{enumerate}

The following sections describe proposed assays and report initial experiments with a simple model system designed to validate their estimations of cryptic sequence complexity.
Software, including full, documented implementations of underlying statistical estimators, is at \url{https://github.com/mmore500/cryptic-sequence-concept} \citep{moreno2024cryptic}.
% This project benefited from availability of many pieces of open-source software \citep{TODO}.
